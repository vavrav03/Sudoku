\documentclass[aspectratio=43]{beamer}
\usetheme{Berlin}

\usepackage[czech]{babel}
\usecolortheme{dolphin}
\usepackage{graphicx}
\usepackage{dirtree}
\usepackage{listings}
\usepackage[T1]{fontenc}
\usepackage{lmodern}
\usepackage[utf8]{inputenc}
\usepackage{caption}
\usepackage{bbding}
\usepackage{xurl}
\usepackage{scrextend}
\usepackage{minted}
\usepackage{appendixnumberbeamer}
\usepackage{numberedblock}


\captionsetup{labelformat=empty}

\beamertemplatenavigationsymbolsempty
\defbeamertemplate*{title page}{customized}[1][]
{
  \usebeamerfont{title}\inserttitle\par
  \usebeamerfont{subtitle}\usebeamercolor[fg]{subtitle}\insertsubtitle\par
  \bigskip
  \usebeamerfont{author}\insertauthor\par
  \usebeamerfont{institute}\insertinstitute\par
  \usebeamerfont{date}\insertdate\par
  \usebeamercolor[fg]{titlegraphic}\inserttitlegraphic
}
\newcommand\hr{\par\vspace{-.5\ht\strutbox}\noindent\hrulefill\par}

\hypersetup{unicode}
\hypersetup{breaklinks=true}


\title{Sudoku se vším všudy}
\subtitle{Ročníkový projekt}
\author{Vladimír Vávra 4.E}
\date{7.března 2022}
\institute{Gymnázium, Praha 6, Arabská 14}
\setbeamertemplate{sidebar right}{}
\setbeamertemplate{footline}{%
  \hfill\textbf{Stránka \insertframenumber{}. z \inserttotalframenumber} 
  \hspace{0.01cm} \vspace{0.1cm}}
\setbeamerfont{footnote}{size=\tiny}

\begin{document}

\begin{frame}[plain]
	\maketitle
\end{frame}

\clearpage
\setcounter{framenumber}{0}

\section{Obecný pohled}
\begin{frame}[fragile]
	\frametitle{O projektu}
	\begin{itemize}
		\item Cíl práce: Vytvořit škálovatelný základ pro stejnojmennou maturitní
		      práci
		\item Aktuální funkce projektu:
		      \begin{itemize}
			      \item Vytvoření uživatelského rozhraní pro hraní klasického sudoku a
			            různých variant
			      \item Registrační systém uživatelů s možností více druhů přihlášení
			      \item Schopnost programu řešit sudoku včetně přidružených variant
			      \item Generátor sudoku různé typy sudoku různých velikostí dle
			            obtížnosti
		      \end{itemize}
	\end{itemize}
\end{frame}

\begin{frame}[fragile]
	\frametitle{Použité technologie}
	\begin{columns}
		\begin{column}{0.6\textwidth}
			\begin{itemize}
				\item MERN Stack
				      \begin{itemize}
					      \item React, Material-UI, SASS -- frontend
					      \item Express + NodeJS -- backend server
					      \item MongoDB -- databáze
				      \end{itemize}
				\item Vývoj:
				      \begin{itemize}
					      \item VS Code
					      \item Git + dropbox -- zálohování kódu
				      \end{itemize}
			\end{itemize}
		\end{column}
		\begin{column}{0.4\textwidth}
			\begin{figure}[H]
				\centering \def\myFigureNameForTeXstudio{../img/mern}
				\includegraphics[width=1\columnwidth]{\myFigureNameForTeXstudio}
			\end{figure}
			\begin{columns}
				\begin{column}{0.3\columnwidth}
					\begin{figure}[H]
						\centering \def\myFigureNameForTeXstudio{../img/vscode}
						\includegraphics[width=1\columnwidth]{\myFigureNameForTeXstudio}
					\end{figure}
				\end{column}

				\begin{column}{0.3\columnwidth}
					\begin{figure}[H]
						\centering \def\myFigureNameForTeXstudio{../img/dropbox}
						\includegraphics[width=1\columnwidth]{\myFigureNameForTeXstudio}
					\end{figure}
				\end{column}
				\begin{column}{0.3\columnwidth}
					\begin{figure}[H]
						\centering \def\myFigureNameForTeXstudio{../img/git}
						\includegraphics[width=1\columnwidth]{\myFigureNameForTeXstudio}
					\end{figure}
				\end{column}
			\end{columns}
		\end{column}
	\end{columns}
\end{frame}

\section{Architektura}
\begin{frame}[fragile]
	\frametitle{Architektura}
	\begin{columns}
		\begin{column}{0.5\columnwidth}
			\begin{itemize}
				\item Uživatel interaguje s frontendem
				\item Frontend se na sudoku dotazuje NodeJS serveru přes REST API
				      (Client side rendering)
				\item NodeJS server vybírá zadání (s řešením) z MongoDB databáze + 
				provede variační
				      práce
				\item Pro vygenerování sudoku -- interakce s CLI na serveru (pro admina)
			\end{itemize}
		\end{column}
		\begin{column}{0.5\columnwidth}
			\begin{figure}[H]
				\centering \def\myFigureNameForTeXstudio{../img/architecture}
				\includegraphics[width=1\columnwidth]{\myFigureNameForTeXstudio}
				\caption{Architektura aplikace}
				\label{fig:architektura}
			\end{figure}
		\end{column}
	\end{columns}
\end{frame}

\begin{frame}[fragile]
	\frametitle{Architektura}
	\begin{itemize}
		\item Frontend
		\begin{itemize}
			\item Atomic design (atoms, molecules, organisms, templates, pages)
			\item Redux state (návrhový vzor kontext)
			\item Další návrhové vzory: při vývoji proxy na server, observer (event 
			listenery)
		\end{itemize}
		\item Backend
		\begin{itemize}
			\item Architektura Boba Martina
			\item Využití návrhových vzorů: Dependency injection (IoC), továrních 
			metod, adaptérů, fasád
			\item Test-driven development -- test před kódem -- 90\% kódu má testy -- 
			unit, integrační
		\end{itemize}
	\end{itemize}
\end{frame}

\section{Autentifikace}
\begin{frame}[fragile]
	\frametitle{Autentifikace}
		\begin{itemize}
			\item Typy:
						\begin{itemize}
							\item Local (heslem)
							\item Google Auth
							\item Facebook Auth
						\end{itemize}
			\item Při stejném emailu možnost přihlášení více providery
			\item Endpoint pro změnu hesla
		\end{itemize}


\end{frame}

\begin{frame}[fragile]
	\frametitle{Ukázka autentifikace}
	
\end{frame}

\section{Typy sudoku}
\begin{frame}[fragile]
	\frametitle{Podporované sudoku}
	\begin{columns}
		\begin{column}{0.5\columnwidth}
			\begin{itemize}
				\item Typy:
				      \begin{itemize}
					      \item Klasické (Classic)
					      \item Diagonální (ClassicX)
					      \item Jigsaw
					      \item Samurai, Samurai Mixed -- není zatím hotové UI
				      \end{itemize}
				\item Velikost:
				      \begin{itemize}
					      \item Pro každý typ individuální
					      \item 4x4, 6x6, 8x8, 9x9, 10x10, 12x12, 14x14, 16x16
				      \end{itemize}
				\item Obtížnost
				      \begin{itemize}
					      \item Easy, normal, hard
					      \item Přímo úměrné počtu chybějících polí -- škálovatelné, 
					      objektivní
				      \end{itemize}
			\end{itemize}
		\end{column}
		\begin{column}{0.5\columnwidth}
			\begin{columns}
				\begin{column}{0.5\columnwidth}
					\begin{figure}[H]
						\centering \def\myFigureNameForTeXstudio{../img/8x8sudoku}
						\includegraphics[width=0.9\columnwidth]{\myFigureNameForTeXstudio}
						\caption{Classic}
						\label{fig:Classic}
					\end{figure}
				\end{column}
				\begin{column}{0.5\columnwidth}
					\begin{figure}[H]
						\centering
						\def\myFigureNameForTeXstudio{../img/8x8DiagSudoku.jpg}
						\includegraphics[width=0.9\columnwidth]{\myFigureNameForTeXstudio}
						\caption{ClassicX}
						\label{fig:ClassicX}
					\end{figure}
				\end{column}
			\end{columns}
			\begin{columns}
				\begin{column}{0.5\columnwidth}
					\begin{figure}[H]
						\centering \def\myFigureNameForTeXstudio{../img/9x9jigsaw.jpg}
						\includegraphics[width=0.9\columnwidth]{\myFigureNameForTeXstudio}
						\caption{Jigsaw}
						\label{fig:Classic}
					\end{figure}
				\end{column}
				\begin{column}{0.5\columnwidth}
					\begin{figure}[H]
						\centering
						\def\myFigureNameForTeXstudio{../img/9x9samurai.png}
						\includegraphics[width=0.9\columnwidth]{\myFigureNameForTeXstudio}
						\caption{Samurai}
						\label{fig:ClassicX}
					\end{figure}
				\end{column}
			\end{columns}
		\end{column}
	\end{columns}
\end{frame}

\begin{frame}[fragile]
	\frametitle{Ukázka her}
	
\end{frame}

\section{Algoritmy}
\begin{frame}[fragile]
	\frametitle{Algoritmus řešení sudoku}

	\begin{itemize}
		\item Backtrackingový algoritmus -- prohledávání do hloubky
		\item Při pokládání čísel kontrola, zda mohu položit -- rekurze / jiné 
		číslo / backtrack
		\item $O(M^{N})$, kde \textit{M} je počet možných dosazovaných čísel a 
		\textit{N} počet políček, za které dosazujeme.
	\end{itemize}
\end{frame}

\begin{frame}[fragile]
	\frametitle{Algoritmus generování sudoku}
		\begin{enumerate}
			\item Vytvoř validní mřížku pro daný typ sudoku 
			\begin{itemize}
				\item Upravený algoritmus řešení
				\item Při přidání políčka je číslo vyškrtnuto z míst, kde nemůže být
				\item Další číslo vybíráno pouze z čísle, které mohou na políčku být
			\end{itemize}
			\item Uber z mřížky N políček (přímo úměrné obtížnosti)
			\item Vyřeš sudoku a zkontroluj, zda má 1 řešení. Má? Ulož ho do 
			databáze. Nemá? Vrať odstraněná políčka a jdi na krok 2.
		\end{enumerate}
\end{frame}

\begin{frame}[fragile]
	\frametitle{Algoritmus na výrobu variant}
	\begin{itemize}
		\item Možnost vytvoření více zadání z jedné vygenerované mřížky s 
		kvadratickou časovou složitostí (v závislosti na velikosti mřížky).
		\item Uživatel nepozná rozdíl, bude-li mít rozestup mezi variantami téhož 
		sudoku
		\item Operace, které můžeme provést jsou:
		\begin{itemize}
		   \item Permutace čísel -- celkem $9!$ možností
		   \item Rotace matice -- 4 možnosti
		   \item Transpozice matice -- dle hlavní diagonály, vedlejší diagonály, 
		   osy x, osy y -- 4 možnosti
		\end{itemize}
		\item $N! * 4 * 4$
		\item pro 9x9 zadání -- 5 806 080 variant.
	\end{itemize}
\end{frame}

\section{Generátor}
\begin{frame}[fragile]
	\frametitle{Generátor}
	\begin{itemize}
		\item CLI rozhraní pro administrátora
		\item Velké mřížky těžké obtížnosti trvají dlouho vygenerovat $\rightarrow$ 
		nutnost manuální kontroly
		\item \texttt{npm run generate}
		\item Nutné zadat:
		\begin{itemize}
			\item Typ sudoku
			\item Velikost mřížky
			\item Obtížnost
			\item Počet zadání tohoto typu
		\end{itemize}
	\end{itemize}
\end{frame}

\begin{frame}[fragile]
	\frametitle{Ukázka generátoru}
\end{frame}

\section{Závěr}
\begin{frame}[fragile]
	\frametitle{Závěr}
	\begin{itemize}
		\item Podařilo se splnit většinu zadání a vytvořit škálovatelný základ pro
		      maturitní práci
		\item Do maturitní práce:
		      \begin{itemize}
			      \item Dokončení rozpracované interní ekonomiky (využití coinů)
			      \item Implementace dalších typů her
			      \item Vylepšení UI pro podporu dalších funkcí (vlastní zadání a jeho
			            vyřešení)
			      \item Bonus: vyřešení z fotografie
		      \end{itemize}
	\end{itemize}
\end{frame}

\appendix
\section{Zdroje}
\begin{frame}[plain]
	\frametitle{Zdroje}
	\begin{itemize}
		\item vscode icon: 
		\url{https://commons.wikimedia.org/wiki/File:Visual_Studio_Code_1.35_icon.svg}
		\item dropbox icon: \url{https://danielgamrot.cz/skvely-tip-pro-dropbox/}
		\item git icon: \url{https://commons.wikimedia.org/wiki/File:Git_icon.svg}
	\end{itemize}
\end{frame}

\end{document}
