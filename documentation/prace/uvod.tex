\chapter*{Úvod}
\pagenumbering{gobble}
\addcontentsline{toc}{chapter}{Úvod}

Zadáním projektu bylo vytvořit webovou aplikaci, jejímž primárním účelem je hraní sudoku v klasické podobě i možných variantách. Aplikace dokáže sudoku vyřešit a vygenerovat zadání. Kromě klasického sudoku dokáže také vygenerovat zadání na základě obtížnosti pro všechny typy, u kterých je aktuálně generování podporováno, čímž se práce rozšířila.

Kromě tohoto také aplikace poskytuje základní funkce vnitřní ekonomiky, jako je možnost různých typů přihlášení a nakupování vylepšení za interní měnu.

Co se použitých technologií píše, frontend byl napsán v knihovně ReactJS, backend v NodeJS s frameworkem Express a jako databáze byla použita MongoDB. V každé z těchto částí je využito mnoha knihoven, které jsou specifikovány v souborech \texttt{package.json}, jenž nacházejí přímo ve zdrojovém kódu.